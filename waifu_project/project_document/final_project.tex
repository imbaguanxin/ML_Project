\documentclass[11.5pt]{article}
\usepackage[margin=1in]{geometry}
\usepackage{hyperref}



\title{Recognize Animation Characters with Machine Learning}

\author{Xin Guan, Ziqian Ge}

\date{}

\begin{document}

\maketitle

\abstract
Please provide a brief abstract of your project.

\vspace{2mm}
\section{Introduction}
As the fandom community grows, a large number of fan-arts made by non-official illustrators began to show up and gradually became an significant part of the community.
Fan-arts spread on the internet are often not labeled or are hard for people to recognize the character it is illustrating, since many anime characters shares similar characteristics and different illustrators may have their own taste.
Untagged illustrations would make some trouble if anyone is trying to make a illustration suggestion system based on the features or characteristics of anime characters, or trying to use GAN (Generative Adversarial Network) to generate fake anime illustrations.
Automatic character recognizing would then make some differences when one is facing a large set of untagged illustrations or would like to add/optimize tags based on character, for example, a character may be bounded to specific tags, like blond, green-eye, sword, armored, etc.\\ \\
This project is aiming to use machine learning to automatically recognize characters in illustrations, based specifically on their faces.

\section{Technical Approach}
Please describe the techniques you have used in order to address the problem. Describe in detail the classification/regression/other techniques you have used in order to tackle the problem.

\section{Experimental Results}
Describe the datasets used for your experiments. Be precise in describing all information about the datasets, including, classes, number of samples per class, features used to represent data, and all pre/post processing of the datasets.\\
Describe the details about the implementation of each algorithm, e.g., how you perform training, validation, testing, values of the hyperparameters and your methods for hyperparameter tuning, training/validation/testing error on the dataset, and all useful plots/tables that help to better interpret your results and your work.

\section{Participants Contribution}
Please list the name of the participants. For each participant explain in details the role he/she played in the project: explain which methods was implemented by which member, which dataset was processed by which member, which experimental results were generated by which members, etc.

\vspace{10mm}
** Please do not change the size of the fonts.

** Please note that your submission must be at most 7 pages long, excluding references.

\end{document}
